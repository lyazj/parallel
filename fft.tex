\documentclass{ctexart}
\usepackage{amsmath,amssymb}

\title{快速傅里叶变换}
\author{未命名作者}

\begin{document}

\maketitle

首先考察连续函数 $f: [0, 2\pi] \to \mathbb R$ 的傅里叶变换。设
\begin{equation}
  f(x) = \sum_{k = -\infty}^{+\infty} a_k\mathrm e^{\mathrm ikx}
\end{equation}
其中的系数 $a_k$ 可以由积分
\begin{equation}
  \int_0^{2\pi}f(x)\mathrm e^{\mathrm -ik^\prime x}\mathrm dx
  =
  \sum_{k = -\infty}^{+\infty} a_k\int_0^{2\pi}\mathrm e^{\mathrm i(k - k^\prime)x}\mathrm dx
  =
  \sum_{k = -\infty}^{+\infty} a_k \cdot 2\pi\delta_{kk^\prime}
  =
  2\pi a_k^{\prime}
\end{equation}
推出:
\begin{equation}
  a_k = \frac{1}{2\pi}\int_0^{2\pi}f(x)\mathrm e^{-ikx}\mathrm dx
\end{equation}
离散化近似得到:
\begin{equation}
  a_k = \frac{1}{n}\sum_{j = 0}^{n - 1}f(2j\pi / n)\mathrm e^{-2\mathrm ijk\pi/n}
\end{equation}
定义:
\begin{equation}
  y_{j,n} = f(2j\pi / n),\ \omega_{k,n} = \mathrm e^{2\mathrm ik\pi / n}
\end{equation}
我们有:
\begin{equation}
  a_k = \frac{1}{n}\sum_{j = 0}^{n - 1}y_{j,n}\omega_{k,n}^{-j}
\end{equation}
再将求和式中的 $k$ 截断至 $0, 1, \dots, n - 1$,即得到 DFT:
\begin{equation}
  y_{j,n} = \sum_{k = 0}^{n - 1}a_{k,n}\omega_{k,n}^j,\
  a_{k,n} = \frac{1}{n}\sum_{j = 0}^{n - 1}y_{j,n}\omega_{k,n}^{-j}
\end{equation}
习惯取 $c_{k,n} = na_{k,n}$,我们有如下关系:
\begin{equation}
  y_{j,n} = \frac{1}{n}\sum_{k = 0}^{n - 1}c_{k,n}\omega_{k,n}^j,\
  c_{k,n} = \sum_{j = 0}^{n - 1}y_{j,n}\omega_{k,n}^{-j}
\end{equation}
验证一下:
\begin{equation}
  \begin{split}
    \frac{1}{n}\sum_{k = 0}^{n - 1}c_{k,n}\omega_{k,n}^j
    &=
    \frac{1}{n}\sum_{k = 0}^{n - 1}\sum_{j^\prime = 0}^{n - 1}y_{j^\prime,n}\omega_{k,n}^{j - j^\prime}
    =
    \frac{1}{n}\sum_{j^\prime = 0}^{n - 1}y_{j^\prime,n}\sum_{k = 0}^{n - 1}\omega_{k,n}^{j - j^\prime}
    \\&=
    \frac{1}{n}\sum_{j^\prime = 0}^{n - 1}y_{j^\prime,n}n\delta_{jj^\prime}
    =
    y_{j,n}
  \end{split}
\end{equation}
这验证了离散傅里叶变换的等式是精确成立的。下面推导各系数 $c_{k,n}$ 之间的递推关系。令 $n \to 2n$,我们有:
\begin{equation}
  \begin{split}
    c_{k,2n}
    &=
    \sum_{j = 0}^{2n - 1}y_{j,2n}\omega_{k,2n}^{-j}
    =
    \sum_{j = 0}^{n - 1}y_{2j,2n}\omega_{k,2n}^{-2j}
    +
    \sum_{j = 0}^{n - 1}y_{2j + 1,2n}\omega_{k,2n}^{-2j - 1}
    \\&=
    \sum_{j = 0}^{n - 1}y_{2j,2n}\omega_{k,n}^{-j}
    +
    \omega_{k,2n}^{-1}\sum_{j = 0}^{n - 1}y_{2j + 1,2n}\omega_{k,n}^{-j}
    \\c_{k + n,2n}
    &=
    \sum_{j = 0}^{2n - 1}y_{j,2n}\omega_{k + n,2n}^{-j}
    =
    \sum_{j = 0}^{n - 1}y_{2j,2n}\omega_{k + n,2n}^{-2j}
    +
    \sum_{j = 0}^{n - 1}y_{2j + 1,2n}\omega_{k + n,2n}^{-2j - 1}
    \\&=
    \sum_{j = 0}^{n - 1}y_{2j,2n}\omega_{k,n}^{-j}
    -
    \omega_{k,2n}^{-1}\sum_{j = 0}^{n - 1}y_{2j + 1,2n}\omega_{k,n}^{-j}
  \end{split}
\end{equation}
观察上式不难看出,$\sum_{j = 0}^{n - 1}y_{2j,2n}\omega_{k,n}^{-j}$ 为 $y_0, y_2, \dots, y_{2n - 2}$ 的第 $k$ 个 DFT 系数,$\sum_{j = 0}^{n - 1}y_{2j + 1,2n}\omega_{k,n}^{-j}$ 为 $y_1, y_3, \dots, y_{2n - 1}$ 的第 $k$ 个 DFT 系数。

下面使用自顶向下的思路推导 FFT 算法。首先,设输入数组的规模为 $n = 2^{N}\ (N \geq 1)$,最后一步状态转移前数组的前后两半部分分别存储上面的两组系数,则最后一步的状态转移方程可以写做:
\begin{equation}
  f(k, N) = \begin{cases}
    f(k, N - 1) + \omega_{k,n}^{-1}f(k + n/2, N - 1), & k < n/2, \\
    f(k - n/2, N - 1) + \omega_{k,n}^{-1}f(k, N - 1), & k \geq n/2. \\
  \end{cases}
\end{equation}
类似地,根据递归原理,第 $M\ (M \geq 1)$ 步的状态转移方程可以写做:
\begin{equation}
  f(k, M) = \begin{cases}
    f(k, M - 1) + \omega_{k,2^M}^{-1}f(k + 2^{M-1}, M - 1), & k\,\%\,2^M < 2^{M-1}, \\
    f(k - 2^{M-1}, M - 1) + \omega_{k,2^M}^{-1}f(k, M - 1), & k\,\%\,2^M \geq 2^{M-1}. \\
  \end{cases}
\end{equation}
下面确定 $y_0, y_1, \dots, y_{n - 1}$ 的初始摆放顺序。在 $N$ 位二进制代数中,指标 $j$ 的最低位决定了 $y_j$ 在最后一次状态转移前只贡献左/右半部分的数组元素,次低位(如有)决定了 $y_j$ 在倒数第二次状态转移前只贡献左/右半部分的左/右半部分的数组元素,以此类推,指标 $j$ 的 $N$ 位 bit swap 值应为它在初始数组中的位置。进一步可以得到,如果初始时 $y$ 值完全按照顺序排列,则将指标 $j$ 与指标 $\mathrm{bitswap}(j, N)$ 处的元素交换一次(不能重复)后,便可得到初始状态 $\{f(j, 0)\}_{j = 0}^{n - 1}$。

\end{document}
